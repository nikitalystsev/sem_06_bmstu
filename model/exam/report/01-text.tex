\section*{Список вопросов для консы} % готов

1. Что подразумевается под оценкой точности в методах РКN, N = 2, 4? Нужно ли писать что то про погрешность методов РК? Нужно ли в вопросе про РК4 добавлять обобщение на систему их 2-х уравнений?

2. Не понял в методе адамса про переход и точки $n$ в точку $n + 1$ за 1 шаг

3. Как правильно выглядит постановка задачи Коши?

\section*{Понятие модели и моделирования. Общая классификация моделей. Требования к моделям. Примеры из конкретных предметных областей.} % готов

\textbf{Моделирование} -- вид человеческой деятельности, при кот. изучение объекта заменяется изучением его модели. Работа с объектом заменяется работой с его моделью. \textbf{Объект} -- процесс, система, явление, событие.

\textbf{Модель} -- некоторый образ этого объекта, ему \textit{не тождественный}.

\textbf{Модель} -- мысленно представляемая или материально реализованная система, кот. отображая и воспроизводя изучаемый объект, замещает его так, что ее изучение позволяет получить новую информацию об объекте.

\textbf{Модель} -- это представление объекта в виде, отличном от формы и способа его реального существования.

\textbf{Классификация моделей (разделяют на 3 больших класса):}

\begin{itemize}[label*=--]
	\item \textit{материальные} (натурные);
	\item \textit{модели суждения:} не будем касаться (вера, религия, философия);
	\item \textit{нематериальные} (абстрактные, идеальные).
\end{itemize}

\textbf{Материальные модели:}

\begin{itemize}[label*=--]
	\item \textit{Физические} -- материальная модель, кот. воспроизводит объект в каком то урезанном виде;
	
	Пример: макет  самолета продувают в аэродинамическую трубу. Весь функционал воспроизводится в точности.
	
	Пример: оптические печи. Исследование воздействия радиационного излучения.
	
	\item \textit{Геометрические} -- воспроизводят форму объекта, с кот. мы работаем;
	
	Пример: макеты архитектурных сооружений, макеты технических устройств, муляжи, манекены, глобусы.
	
	\item \textit{Аналоговые}
	
	Бывает так, что одни и те же явления имеют подобия. Это подобие не обнаруживается, если смотреть внешне. Но, описывая эти явления или процессы, можно обнаружить, что 
	уравнения, кот. их описывают, похожие.
	
	Пример: хочу определить температурное поле в какой то сложной пластине с какими то сложными краевыми условиями. Чтобы определить температурное поле в этой ситуации, можно построить эл. цепь по определенным правилам, в кот. будут сопротивления, конденсаторы, катушки и т. д.
	Измеряя сопротивление отдельных участков этой цепи, напряжение опред-х участков, токи, кот. текут, можно сопоставить этим электрическим полям температурное поле, тепловые потоки, кот. распр-ся в этой пластине, или, например, концентрацию частиц (опис-ся теми же самыми ур-ми), диффузию.
	
	На это принципе построены аналоговые ЭВМ.
\end{itemize}


\textbf{Нематериальные модели:}

Классификация м.б. проведена по разным признакам (перечень признаков неполный): 

Одни и те же модели м.б. отнесены к разным классам.

\begin{itemize}[label*=--]
	\item \textit{по форме выражения} -- механические, логические, математические;
	\item \textit{по предмету исследования} -- физические, химические, технические, медицинские, физиологические и т. д.;
	\item \textit{по природе явления} -- социальные, экономические, биологические, психологические, молекулярные, квантовые;
	\item \textit{по степени точности} -- приближенные, точные, достоверные и т. д.;
	\item \textit{по задачам исследования} -- эвристические, прогностические;
	\item \textit{по способы выражения} -- графические, текстовые, символьные;
	\item \textit{по свойствам отражения} -- функциональные, информационные, системные.
\end{itemize}


\textbf{Требования к математическим моделям:}

\begin{itemize}[label*=--]
	\item \textit{адекватность} -- соответствие модели задачам, кот. стоят перед моделью (в какой степени она удовлетворяет требованиям модели);
	
	\item \textit{точность};
	
	\item \textit{универсальность} -- модель описывает не одно явление, а целый класс явлений;
	
	В физике одни и те же уравнения описывают распр-е тепла, диффузию частиц и др. процессы.
	
	\item \textit{Экономичность}.
\end{itemize}


\section*{Схема вычислительного эксперимента.} % готов

\includeimage
{схема} % Имя файла без расширения (файл должен быть расположен в директории inc/img/)
{f} % Обтекание (без обтекания)
{h} % Положение рисунка (см. figure из пакета float)
{1\textwidth} % Ширина рисунка
{Схема математического моделирования} % Подпись рисунка

Ошибка в модели -- самое неприятное.

Программа -- экспериментальная установка.

Процедура проведения численных расчетов с помощью программы называется \textbf{вычислительным экспериментом}. Он не уступает в точности натурному эксперименту. Он может заменить натурный эксперимент.

\section*{Понятие математической модели. Функции моделей. Источники погрешностей при построении модели, алгоритмизации и программировании.} % готов

\textbf{Математическая модель} -- представление объекта в виде мат. объектов (формул, уравнений разного типа, логических соотношений), т.е. с использованием математического аппарата.

\textbf{Математическое моделирование} -- замена реального объекта его мат. моделью и изучение в дальнейшем построенной модели вместо изучаемого объекта.

Результаты распр-ся на тот объект, который мы изучаем.


\textbf{Классификация мат. моделей:}

\begin{itemize}[label*=--]
	\item \textit{регулярные (функциональные)} -- позволяют описать функционирование того объекта, с кот. мы работаем;
	
	\item \textit{модели идентификации} -- модели черного ящика -- строятся по принципу <<вход выход>>;
	
	Есть некий (м.б. материальный объект). Задаем на вход какие то хар-ки и наблюдаем за выходом. Устанавливаем связь между входом и выходом. Как работает объект не знаем.
	
	Если моделируем систему, то сначала строим модели отдельных элементов и объединяем в систему. Элементы между собой обмениваются какой то информацией. Удобно построить модель идентификации, т.е. она м.б. построена не только когда есть реальный объект.
	
	Модель идентификации, по сути, формула, вход-выход.
	
	С другой стороны, выходом м.б. вектор, входом тоже м.б. вектор. Связь устанавливается эмпирически. Но связь можно устанавливать, строя регулярные модели, проводя численные эксперименты уже над моделями, и строить модель идентификации. Тогда она в себе будет суммировать результаты применения этих регулярных моделей. 
	
	Когда строим модель системы, то, построив модель идентификации отдельных элементов (а это формулы). С пом. этих формул мы обеспечиваем взаимодействие этих элементов и функционирование системы в целом. 
	
	Сами по себе применяются сотни лет. Это наиболее простой и понятный способ, когда мы не знаем, как это все работает, но мы знаем, как откликается тот или иной объект на воздействие на него. 
	
	Имеют важное значение при исследование систем.
	
	\item \textit{имитационные};
	
	Работают с вероятностями. 
	
	Пример: теория массового обслуживания, теория очередей.
	
	Применяются и в физических задачах (многие имеют случайный характер). 
	
	\item \textit{феноменологические} -- позволяют описать внешнюю сторону;
	
	Пример: термодинамика -- она не устанавливает причины тех или иных явлений. Она говорит, что если сжал газ, то давление и температура будет повышаться. А почему так, она не дает ответа. Эта наука строит феноменологические модели. 
	
\end{itemize}


\textbf{Важные функции математических моделей:}

\begin{itemize}[label*=--]
	\item \textit{средство познания действительности. }
	
	Уже на этапе создания модели выясняются нелогичности и пробелы в наших знаниях, формулируются направления необходимых исследований, конкретизируются соответствующие задачи. Вся научная деятельность состоит в построении и исследовании моделей.
	
	\item \textit{средство накопления и передачи информации и общения, благодаря компактности,
	точности и объективности модели.}
	
	\item \textit{средство обучения и тренажа, способствующее приобретению профессиональных
	навыков без риска для жизни и здоровья.}
	
	\item \textit{средство прогнозирования поведения объекта. }
	
	Математическое моделирование позволяет еще до возникновения реальной ситуации оценить условия ее возникновения и способы управления развитием событий, выбрать оптимальные параметры и режимы работы системы до ее реального создания. Появляется возможность исследовать последствия катастроф, наступление которых невозможно допустить (взрыв ядерной установки, космические катастрофы, отравление океана, глобальные изменения климата и т.д.).
\end{itemize}

\textbf{При моделировании возникает 4 вида погрешности:}

\begin{itemize}[label*=--]
	\item \textit{погрешность исходных данных} -- неустранимая, принимается как данность;
	
	\item \textit{погрешность модели} -- делается с нек-м приближением. Зависит от того, насколько вы специалист в той предметной области, кот. изучаем;
	
	\item \textit{погрешность метода} -- метода реализации модели. Заботиться о том, чтобы метод обеспечивал нужную для задачи точность;
	
	\item \textit{погрешность округления} -- погрешность, вносимая компьютерными вычислениями (представление чисел, погрешность, связанная с операциями);
	
	Нарастает как квадратный корень из количества операций. Погрешность метода д.б. раз в 5 меньше, чем погрешность округления.
\end{itemize}


\section*{Понятие корректности постановки задач. Привести примеры некорректно поставленных и слабо обусловленных задач и неустойчивых алгоритмов.} % готов

Задача считается корректно поставленной, если ее решение существует, единственно и устойчиво ко входным данным.

Устойчивость означает, что малое изменение входных данных должно приводить к малому изменению выходных данных, т.е. должно иметь место непрерывная зависимость выходных параметров от входных данных.

\includeimage
{устойчивость} % Имя файла без расширения (файл должен быть расположен в директории inc/img/)
{f} % Обтекание (без обтекания)
{h} % Положение рисунка (см. figure из пакета float)
{1\textwidth} % Ширина рисунка
{Устойчивость решения задачи} % Подпись рисунка

Если задача неустойчива, то решать ее обычными способами сложно. На любом этапе могут появиться погрешности. Это означает, что малейшие погрешности приведут к тому, что ошибка в ходе вычислений будет нарастать.

Пример: обратные задачи. Прямая задача: задан поток и прочее -- определить температурное поле. Обратная задача: по температурному полю найти заданный поток. Обращая время вспять и зная текущее состояние поля найти начальное поле, от которого стартовал. \textbf{Это некорректные задачи.}

Для таких задач придуманы свои методы решения -- \textbf{регуляризация}. Идея: превращаем задачу в корректную, введя некий параметр и решаем по опр. алгоритмам, а потом, устремляя параметр к нулю, переходим к решению некорректной задачи. 

\includeimage
{слабо_уст} % Имя файла без расширения (файл должен быть расположен в директории inc/img/)
{f} % Обтекание (без обтекания)
{h} % Положение рисунка (см. figure из пакета float)
{1\textwidth} % Ширина рисунка
{Слабоустойчивая задача}

Такие задачи называются слабоустойчивыми (плохо обусловленными).

Пример: задачи линейной алгебры, в частности, решение СЛАУ
 
\includeimage
{пример_слабо_уст} % Имя файла без расширения (файл должен быть расположен в директории inc/img/)
{f} % Обтекание (без обтекания)
{h} % Положение рисунка (см. figure из пакета float)
{0.7\textwidth} % Ширина рисунка
{Пример слабоустойчивой задачи}

помимо неустойчивых задач, неустойчивым м.б. алгоритм, выбранный для решения.

\includeimage
{пример_слабо_уст_алг} % Имя файла без расширения (файл должен быть расположен в директории inc/img/)
{f} % Обтекание (без обтекания)
{h} % Положение рисунка (см. figure из пакета float)
{1\textwidth} % Ширина рисунка
{Пример слабоустойчивого алгоритма (задача устойчива)}

\clearpage

\section*{Общая классификация методов построения математических моделей.} 

% в процессе

\section*{ Построение математических моделей на основе законов природы. Привести примеры.} 

% в процессе

\section*{Построение математических моделей на основе вариационных принципов. Привести примеры.} 

% в процессе

\section*{Построение математических моделей выстраиванием иерархии сверху - вниз и снизу - вверх. Привести примеры.} 

% в процессе

\section*{Построение математических моделей методом аналогий. Привести примеры.} 

% в процессе

\section*{Понятие ОДУ. Сведение ОДУ произвольного порядка к системе ОДУ первого порядка. Привести примеры.} % готов 

\includeimage
{ОДУ} % Имя файла без расширения (файл должен быть расположен в директории inc/img/)
{f} % Обтекание (без обтекания)
{h} % Положение рисунка (см. figure из пакета float)
{1\textwidth} % Ширина рисунка
{ОДУ}

\includeimage
{ОДУ2} % Имя файла без расширения (файл должен быть расположен в директории inc/img/)
{f} % Обтекание (без обтекания)
{h} % Положение рисунка (см. figure из пакета float)
{1\textwidth} % Ширина рисунка
{ОДУ2}

\clearpage

Уравнение в частных производных содержит минимум 2 независимых переменных.

\includeimage
{свед_к_системе} % Имя файла без расширения (файл должен быть расположен в директории inc/img/)
{f} % Обтекание (без обтекания)
{h} % Положение рисунка (см. figure из пакета float)
{1\textwidth} % Ширина рисунка
{Пример сведения ОДУ 2-го порядка к системе уравнеиний 1-го порядка}

\section*{Постановки задачи Коши и краевой задачи для ОДУ.} % готов

\includeimage
{постановка_задачи_коши} % Имя файла без расширения (файл должен быть расположен в директории inc/img/)
{f} % Обтекание (без обтекания)
{h} % Положение рисунка (см. figure из пакета float)
{0.7\textwidth} % Ширина рисунка
{Постановка задачи Коши}

$a \leq x \leq b$

\textbf{Про краевую задачу:}

В отличие от задачи Коши, где условие ставится на одном крае, здесь м.б. поставлено в нескольких точках, по крайней мере в двух. два условия позволят найти 2 константы => уравнение д.б. минимум 2-го порядка или 2 уравнения 1-го порядка. Дифф. ур-е 4-го порядка потребует 4 условия.


\includeimage
{постановка_краевой_задачи} % Имя файла без расширения (файл должен быть расположен в директории inc/img/)
{f} % Обтекание (без обтекания)
{h} % Положение рисунка (см. figure из пакета float)
{1\textwidth} % Ширина рисунка
{Постановка краевой задачи}

$a \leq x \leq b$

$a \leq \xi_k \leq b$

\section*{Метод Пикара в задаче Коши для ОДУ. Привести пример.} % готов

Если правую часть рассматривать как функцию от 2-х переменных, то это целая плоскость. Давайте правую часть рассматривать на решении этого уравнения. Тогда правая часть зависит только от $x$

\clearpage

\includeimage
{метод_пикара} % Имя файла без расширения (файл должен быть расположен в директории inc/img/)
{f} % Обтекание (без обтекания)
{h} % Положение рисунка (см. figure из пакета float)
{0.7\textwidth} % Ширина рисунка
{Метод Пикара}

Простой пример: 

\includeimage
{пример_метод_пикара} % Имя файла без расширения (файл должен быть расположен в директории inc/img/)
{f} % Обтекание (без обтекания)
{h} % Положение рисунка (см. figure из пакета float)
{1\textwidth} % Ширина рисунка
{Простой пример применения метода Пикара}

Сложный пример: 

\includeimage
{сложн_пример_метод_пикара} % Имя файла без расширения (файл должен быть расположен в директории inc/img/)
{f} % Обтекание (без обтекания)
{h} % Положение рисунка (см. figure из пакета float)
{\textwidth} % Ширина рисунка
{Сложный пример применения метода Пикара}

\clearpage

\section*{Метод Рунге - Кутта 2-го порядка точности в задаче Коши для ОДУ. Оценка точности.} % готов, но вопросы есть


\includeimage
{рк2_часть1} % Имя файла без расширения (файл должен быть расположен в директории inc/img/)
{f} % Обтекание (без обтекания)
{h} % Положение рисунка (см. figure из пакета float)
{\textwidth} % Ширина рисунка
{РК2 (часть 1)}

\includeimage
{рк2_часть2} % Имя файла без расширения (файл должен быть расположен в директории inc/img/)
{f} % Обтекание (без обтекания)
{h} % Положение рисунка (см. figure из пакета float)
{\textwidth} % Ширина рисунка
{РК2 (часть 2)}

\includeimage
{рк2_часть3} % Имя файла без расширения (файл должен быть расположен в директории inc/img/)
{f} % Обтекание (без обтекания)
{h} % Положение рисунка (см. figure из пакета float)
{\textwidth} % Ширина рисунка
{РК2 (часть 3)}

\includeimage
{рк2_часть4} % Имя файла без расширения (файл должен быть расположен в директории inc/img/)
{f} % Обтекание (без обтекания)
{h} % Положение рисунка (см. figure из пакета float)
{\textwidth} % Ширина рисунка
{РК2 (часть 4)}

\section*{Метод Рунге - Кутта 4-го порядка точности в задаче Коши для ОДУ. Оценка точности.} % готов, но вопросы есть

\includeimage
{рк4_часть1} % Имя файла без расширения (файл должен быть расположен в директории inc/img/)
{f} % Обтекание (без обтекания)
{h} % Положение рисунка (см. figure из пакета float)
{\textwidth} % Ширина рисунка
{РК4 (часть 1)}

\includeimage
{рк4_часть2} % Имя файла без расширения (файл должен быть расположен в директории inc/img/)
{f} % Обтекание (без обтекания)
{h} % Положение рисунка (см. figure из пакета float)
{\textwidth} % Ширина рисунка
{РК4 (часть 2)}

\section*{Метод Адамса в задаче Коши для ОДУ.} 

Очень популярный; много модификаций. Применялся для расчетов баллистических снарядов.

Это представитель класса многошаговых методов.

Получено решение в трех точках. В четвертой используется решение в этих трех точках, а не одно предыдущее.

Если правую часть рассматривать на интегральной кривой, то это будет функция от $x$.

\includeimage
{метод_адамса} % Имя файла без расширения (файл должен быть расположен в директории inc/img/)
{f} % Обтекание (без обтекания)
{h} % Положение рисунка (см. figure из пакета float)
{\textwidth} % Ширина рисунка
{Метод Адамса}


\section*{Неявные численные методы (Эйлера, трапеций, Гира) в задаче Коши для ОДУ.} 