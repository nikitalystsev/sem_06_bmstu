\chapter{Выполнение задания}

\section{Средства реализации}

В качестве языка программирования был выбран $C++$ \cite{pl}.

\section{Код программы}

В листинге \ref{lst:codeForHWPart1.txt} и \ref{lst:codeForHWPart2.txt}  представлена реализация алгоритма составления файла словаря с количеством употреблений каждой $N$-граммы букв из одного слова в тексте на русском языке.

\includelisting
{codeForHWPart1.txt} % Имя файла с расширением (файл должен быть расположен в директории inc/lst/)
{Реализация алгоритма составления файла словаря с количеством употреблений каждой $N$-граммы букв из одного слова в тексте на русском языке (начало)} % Подпись листинга

\includelisting
{codeForHWPart2.txt} % Имя файла с расширением (файл должен быть расположен в директории inc/lst/)
{Реализация алгоритма составления файла словаря с количеством употреблений каждой $N$-граммы букв из одного слова в тексте на русском языке (конец)} % Подпись листинга 


\section{Графовые модели программы}

На рисунке \ref{img:controlGraph} представлен граф управления.

\includeimage
{controlGraph} % Имя файла без расширения (файл должен быть расположен в директории inc/img/)
{f} % Обтекание (без обтекания)
{h} % Положение рисунка (см. figure из пакета float)
{0.7\textwidth} % Ширина рисунка
{Граф управления} % Подпись рисунка

\clearpage

На рисунке \ref{img:infoGraph} представлен информационный граф.

\includeimage
{infoGraph} % Имя файла без расширения (файл должен быть расположен в директории inc/img/)
{f} % Обтекание (без обтекания)
{h} % Положение рисунка (см. figure из пакета float)
{0.7\textwidth} % Ширина рисунка
{Информационный граф} % Подпись рисунка

\clearpage

На рисунках \ref{img:operHistPart1} и \ref{img:operHistPart2} представлена операционная история.

\includeimage
{operHistPart1} % Имя файла без расширения (файл должен быть расположен в директории inc/img/)
{f} % Обтекание (без обтекания)
{h} % Положение рисунка (см. figure из пакета float)
{0.7\textwidth} % Ширина рисунка
{Операционная история (начало)} % Подпись рисунка

\includeimage
{operHistPart2} % Имя файла без расширения (файл должен быть расположен в директории inc/img/)
{f} % Обтекание (без обтекания)
{h} % Положение рисунка (см. figure из пакета float)
{0.7\textwidth} % Ширина рисунка
{Операционная история (конец)} % Подпись рисунка

\clearpage

На рисунках \ref{img:infoHistPart1} и \ref{img:infoHistPart2} представлена информационная история.

\includeimage
{infoHistPart1} % Имя файла без расширения (файл должен быть расположен в директории inc/img/)
{f} % Обтекание (без обтекания)
{h} % Положение рисунка (см. figure из пакета float)
{0.7\textwidth} % Ширина рисунка
{Информационная история (начало)} % Подпись рисунка

\includeimage
{infoHistPart2} % Имя файла без расширения (файл должен быть расположен в директории inc/img/)
{f} % Обтекание (без обтекания)
{h} % Положение рисунка (см. figure из пакета float)
{0.7\textwidth} % Ширина рисунка
{Информационная история (конец)} % Подпись рисунка

\clearpage

\section{Возможность распараллеливания}

В качестве способа распараллеливания можно разделить строки файла между потоками и запустить обработку частей текста в отдельных потоках, а затем объединить результаты.
