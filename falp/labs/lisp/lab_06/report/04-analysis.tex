\chapter{Практические задания}

\section{Задание 1}

Написать хвостовую рекурсивную функцию my-reverse, которая развернет верхний уровень своего списка-аргумента lst.

\includelisting
{task-01.txt} % Имя файла с расширением (файл должен быть расположен в директории inc/lst/)
{Решение задания №1} % Подпись листинга

\section{Задание 2}

Написать функцию, которая возвращает первый элемент списка-аргумента, который сам является непустым списком.

\includelisting
{task-02.txt} % Имя файла с расширением (файл должен быть расположен в директории inc/lst/)
{Решение задания №2} % Подпись листинга


\section{Задание 3}

Написать функцию, которая выбирает из заданного списка только те числа, которые больше 1 и меньше 10. (Вариант: между двумя заданными границами.)

\includelisting
{task-03.txt} % Имя файла с расширением (файл должен быть расположен в директории inc/lst/)
{Решение задания №3} % Подпись листинга

\section{Задание 4}

Напишите рекурсивную функцию, которая умножает на заданное число-аргумент все числа из заданного списка-аргумента, когда

\begin{enumerate}[label={\alph*)}]
	\item все элементы списка --- числа,
	\item элементы списка --- любые объекты.
\end{enumerate}

\includelisting
{task-04.txt} % Имя файла с расширением (файл должен быть расположен в директории inc/lst/)
{Решение задания №4} % Подпись листинга


\section{Задание 5}

Напишите функцию, select-between, которая из списка-аргумента, содержащего только числа, выбирает только те, которые расположены между двумя указанными границами-аргументами и возвращает их в виде списка (упорядоченного по возрастанию списка чисел (+ 2 балла)).

\includelisting
{task-05.txt} % Имя файла с расширением (файл должен быть расположен в директории inc/lst/)
{Решение задания №5} % Подпись листинга

\section{Задание 6}

Написать рекурсивную версию (с именем rec-add) вычисления суммы чисел заданного списка:

\begin{enumerate}[label={\alph*)}]
	\item одноуровнего смешанного;
	\item структурированного.
\end{enumerate}

\includelisting
{task-06.txt} % Имя файла с расширением (файл должен быть расположен в директории inc/lst/)
{Решение задания №6} % Подпись листинга

\section{Задание 7}

Написать рекурсивную версию с именем recnth функции nth.

\includelisting
{task-07.txt} % Имя файла с расширением (файл должен быть расположен в директории inc/lst/)
{Решение задания №7} % Подпись листинга

\clearpage

\section{Задание 8}

Написать рекурсивную функцию allodd, которая возвращает t когда все элементы списка нечетные.

\includelisting
{task-08.txt} % Имя файла с расширением (файл должен быть расположен в директории inc/lst/)
{Решение задания №8} % Подпись листинга


\section{Задание 9}

Написать рекурсивную функцию, которая возвращает первое нечетное число из списка (структурированного), возможно создавая некоторые вспомогательные функции.

\includelisting
{task-09.txt} % Имя файла с расширением (файл должен быть расположен в директории inc/lst/)
{Решение задания №9} % Подпись листинга

\section{Задание 10}

Используя cons-дополняемую рекурсию с одним тестом завершения, написать функцию которая получает как аргумент список чисел, а возвращает список квадратов этих чисел в том же порядке.

\includelisting
{task-10.txt} % Имя файла с расширением (файл должен быть расположен в директории inc/lst/)
{Решение задания №10} % Подпись листинга