\chapter{Практические задания}

\section{Задание 1}

Чем принципиально отличаются функции \texttt{cons, list, append}?

Пусть \texttt{(setf lst1 '(a b)) (setf lst2 '(c d))}.
Каковы результаты вычисления следующих выражений?

\includelisting
{task-01.txt} % Имя файла с расширением (файл должен быть расположен в директории inc/lst/)
{Решение задания №1} % Подпись листинга


\section{Задание 2}

Каковы результаты вычисления следующих выражений, и почему?

\includelisting
{task-02.txt} % Имя файла с расширением (файл должен быть расположен в директории inc/lst/)
{Решение задания №2} % Подпись листинга

\clearpage

\section{Задание 3}

Написать, по крайней мере, два варианта функции, которая возвращает последний элемент своего списка-аргумента.

\includelisting
{task-03.txt} % Имя файла с расширением (файл должен быть расположен в директории inc/lst/)
{Решение задания №3} % Подпись листинга

\section{Задание 4}

Написать, по крайней мере, два варианта функции, которая возвращает свой список-аргумент без последнего элемента.

\includelisting
{task-04.txt} % Имя файла с расширением (файл должен быть расположен в директории inc/lst/)
{Решение задания №4} % Подпись листинга

\section{Задание 5}

Напишите функцию \texttt{swap-first-last}, которая переставляет в списке-аргументе первый и последний элементы

\includelisting
{task-05.txt} % Имя файла с расширением (файл должен быть расположен в директории inc/lst/)
{Решение задания №5} % Подпись листинга

\section{Задание 6}

Написать простой вариант игры в кости, в котором бросаются две правильные кости. Если сумма выпавших очков равна 7 или 11 --- выигрыш, если выпало (1,1) или (6,6) --- игрок получает право снова бросить кости, во всех остальных случаях ход переходит ко второму игроку, но запоминается сумма выпавших очков. Если второй игрок не выигрывает абсолютно, то выигрывает тот игрок, у которого больше очков. Результат игры и значения выпавших костей выводить на экран с помощью функции \texttt{print}.

\includelisting
{task-06-part1.txt} % Имя файла с расширением (файл должен быть расположен в директории inc/lst/)
{Решение задания №6 (начало)} % Подпись листинга

\clearpage

\includelisting
{task-06-part2.txt} % Имя файла с расширением (файл должен быть расположен в директории inc/lst/)
{Решение задания №6 (конец)} % Подпись листинга

\section{Задание 7}

Написать функцию, которая по своему списку-аргументу lst определяет, является ли он палиндромом (то есть равны ли lst и (reverse lst)).

\includelisting
{task-07.txt} % Имя файла с расширением (файл должен быть расположен в директории inc/lst/)
{Решение задания №7} % Подпись листинга

\section{Задание 8}

Напишите свои необходимые функции, которые обрабатывают таблицу из 4-х точечных пар: (страна . столица), и возвращают по стране - столицу, а по столице - страну.

\clearpage

\includelisting
{task-08.txt} % Имя файла с расширением (файл должен быть расположен в директории inc/lst/)
{Решение задания №8} % Подпись листинга

\section{Задание 9}

9. Напишите функцию, которая умножает на заданное число-аргумент первый числовой элемент списка из заданного 3-х элементного списка-аргумента, когда а) все элементы списка - числа, б) элементы списка - любые объекты.

\begin{enumerate}[label={\alph*)}]
	\item все элементы списка - числа,
	\item элементы списка - любые объекты.
\end{enumerate}

\includelisting
{task-09.txt} % Имя файла с расширением (файл должен быть расположен в директории inc/lst/)
{Решение задания №9} % Подпись листинга