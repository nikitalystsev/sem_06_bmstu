\chapter{Ответы на теоретические вопросы к лабораторной работе}

\section{Элементы языка: определение, синтаксис, представление в памяти}

\subsection{Определение}

Вся информация (данные и программы) в Lisp представляются в виде символьных выражений --- S-выражений. По определению:

\begin{center}
	\captionsetup{justification=raggedright,singlelinecheck=off}
	\begin{lstlisting}
		S-выражение ::= <атом> | <точечная пара>
	\end{lstlisting}
\end{center}	

\textbf{Атомы} могут быть следующими.

\begin{enumerate}[label={\arabic*)}]
	\item Символы (идентификаторы) --- синтаксически представляется как набор букв и цифр, начинающийся с буквы.
	\item Специальные символы --- T, Nil (используются для обозначения логи- ческих констант).
	\begin{itemize}[label*=--]
		\item T --- обозначает логическое значение <<Истина>>, истинным значением является все, отличное от Nil.
		\item Nil --- обозначает логическое значение <<Ложь>>, также обозначает пустой список.
	\end{itemize}
	\item Самоопределимые атомы -- натуральные числа, дробные числа, веще- ственные числа, строки --- последовательность символов, заключенных в двойные апострофы (например "abc").
\end{enumerate}

\textbf{Точечная пара} --- (A.B). Cтроится с помощью бинарного узла.

\begin{center}
	\captionsetup{justification=raggedright,singlelinecheck=off}
	\begin{lstlisting}
		Точечная пара ::= (<атом>.<атом>) | 
		(<атом>.<точечная пара>) |
		(<точечная пара>.<атом>) | 
		(<точечная пара>.<точечная пара>)
	\end{lstlisting}
\end{center}	

\textbf{Список} --- динамическая структура данных, которая может быть пустой или непустой. Если она не пустая, то состоит из двух элементов:

\begin{enumerate}[label={\arabic*)}]
	\item голова --- любая структура;
	\item хвост --- список.
\end{enumerate}

\begin{center}
	\captionsetup{justification=raggedright,singlelinecheck=off}
	\begin{lstlisting}
		Список ::= <пустой список> | <непустой список>, где 
		<пустой список> ::= () | Nil ,
		<непустой список> ::= (<первый элемент>.<хвост>), 
		<первый элемент> ::= <S-выражение>,
		<хвост> ::= <список>.
	\end{lstlisting}
\end{center}	

\subsection{Синтаксис}

Любая структура (точечная пара или список) заключается в круглые
скобки (A.B) --- точечная пара, (А) --- список из одного элемента, пустой список изображается как Nil или ().

Непустой список можно записать следующими образами: (А.(B.(C.(D())))) или (A B C D).

Элементы списка могут, в свою очередь, быть списками (любой список за- ключается в круглые скобки), например — (А (B C) (D (E))). Таким образом, синтаксически наличие скобок является признаком структуры — списка или точечной пары.

\subsection{Представление в памяти}

Любая непустая структура Lisp в памяти представляется списковой ячейкой, хранящей два указателя: на голову и хвост.

\clearpage

\begin{enumerate}[label={\arabic*)}]
	\item (A.B) --- точечная пара.
	
	\includeimage
	{picture1} % Имя файла без расширения (файл должен быть расположен в директории inc/img/)
	{f} % Обтекание (без обтекания)
	{h} % Положение рисунка (см. figure из пакета float)
	{0.3\textwidth} % Ширина рисунка
	{Представление в памяти (A.B)} % Подпись рисунка
	 
	\item (A B) — список из двух элементов.
	
	\includeimage
	{picture2} % Имя файла без расширения (файл должен быть расположен в директории inc/img/)
	{f} % Обтекание (без обтекания)
	{h} % Положение рисунка (см. figure из пакета float)
	{0.5\textwidth} % Ширина рисунка
	{Представление в памяти (A B)} % Подпись рисунка
	
\end{enumerate}


\section{Особенности языка Lisp. Структура программы. Символ апостроф}

Особенности языка Lisp следующие:

\begin{enumerate} [label={\arabic*)}]
	\item в Lisp используется символьная обработка.
	\item программа может быть представлена в виде данных, поэтому она может изменять сама себя.
	\item Lisp является бестиповым языком, так как он работает только на указателях.
	\item память выделяется блоками. LISP сам распределяет память.
	\item программа и данные в LISP представлены списками.
\end{enumerate}

Символ апостроф (<<'>>) — блокирует вычисление своего аргумента. В качестве своего значения выдает сам аргумент, не вычисляя его. Перед константами --- числами и атомами T и Nil --- апостроф можно не ставить.

\section{Базис языка Lisp. Ядро языка}

\textbf{Базис языка} --- минимальный набор конструкций языка и структур данных, с помощью которых можно решить любую задачу.

Базис языка Lisp состоит из:

\begin{enumerate}[label={\arabic*)}]
	\item структур, атомов;
	\item примитивных функций (car, cdr);
	\item специальных функций, управляющих обработкой структур, представляющих вычислимые выражения (quote).
\end{enumerate}

\textbf{Ядро} --- основные действия, которые наиболее часто используются. Ядро шире, чем базис.