\chapter{Практические задания}

\section{Задание}

Разработать свою программу~--- <<Телефонный справочник и автомобили>>. 

<<Телефонный справочник>>:
\begin{itemize}[label*=--]
	\item Фамилия;
	\item №тел.;
	\item Адрес(Город, Улица, №дома, №кв.);
\end{itemize}

<<Автомобили>>:
\begin{itemize}[label*=--]
	\item Фамилия владельца;
	\item Марка;
	\item Цвет;
	\item Стоимость;
	\item Номер.
\end{itemize}

\section{Текст программы}

\includelisting
{prog-part1.txt} % Имя файла с расширением (файл должен быть расположен в директории inc/lst/)
{Текст программы (начало)} % Подпись листинга

\includelisting
{prog-part2.txt} % Имя файла с расширением (файл должен быть расположен в директории inc/lst/)
{Текст программы (конец)} % Подпись листинга
	
\section{Вопросы}

Существует ли человек с фамилией <<Sidorov>>, проживающий в городе <<Arkhangelsk>>?
	
\includelisting
{question1.txt} % Имя файла с расширением (файл должен быть расположен в директории inc/lst/)
{Вопрос №1 и ответ на него} % Подпись листинга

Существуют ли люди, проживающие в городе <<SaintPetersburg>>? Если существуют, то с какими фамилиями?

\includelisting
{question2.txt} % Имя файла с расширением (файл должен быть расположен в директории inc/lst/)
{Вопрос №2 и ответ на него} % Подпись листинга

Автомобилями какой марки владеет человек с фамилией <<Lystsev>>?

\includelisting
{question3.txt} % Имя файла с расширением (файл должен быть расположен в директории inc/lst/)
{Вопрос №3 и ответ на него} % Подпись листинга

Существуют ли люди, владеющие автомобилем цвета <<burgundy>>? Если существуют, вывести фамилию, марку и номер машины.

\includelisting
{question4.txt} % Имя файла с расширением (файл должен быть расположен в директории inc/lst/)
{Вопрос №4 и ответ на него} % Подпись листинга

\clearpage

Существуют ли люди, владеющие автомобилем марки <<Mazda>> цвета <<black>>? Если существуют, вывести фамилию, город и номер телефона человека.


\includelisting
{question5.txt} % Имя файла с расширением (файл должен быть расположен в директории inc/lst/)
{Вопрос №5 и ответ на него} % Подпись листинга

\section*{Вывод}

\begin{enumerate}
	\item \textbf{Что собой представляет программа <<Телефонный справочник>> на Prolog?}
	
	Она представляет собой базу знаний, состоящую предложений. Предложения включают в себя факты <<person>> и <<car>> и конъюктивное правило <<search\_rule>>.
	
	\item \textbf{Какова ее структура?}
	
	\begin{itemize}[label*=--]
		\item \textbf{DOMAINS}~--- раздел для описания доменов;
		\item \textbf{PREDICATES}~--- раздел для описания предикатов;
		\item \textbf{CLAUSES}~--- раздел для описания предложений базы знаний;
		\item \textbf{GOAL}~--- раздел для описания цели (вопроса).
	\end{itemize}
	
	\item \textbf{Как формируется результат работы программы?}
	
	Система в процессе поиска подбирает ответ (<<да>>, <<нет>>) и значения, дающие ответ <<да>>. В ходе получения ответа система подбирает знания, то есть сопоставляет термы.
	
\end{enumerate}