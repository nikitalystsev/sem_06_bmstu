\chapter{Теоретические сведения}

\section{Определение $\gamma$-доверительного интервала для значения параметра распределения случайной величины}

Пусть $X$ --- случайная величина, закон распределения которой известен с
точностью до неизвестного параметра $\theta$.

\textbf{Опр.} Интервальной оценкой c уровнем доверия $\gamma \in (0, 1)$ параметра $\theta$ называется пара статистик $\underline \theta (\vec X)$ и $\overline \theta (\vec X)$ таких, что выполняется равенство:

$$ 
	 P\{ \underline \theta (\vec X) < \theta  < \overline \theta (\vec X)\} = \gamma.
$$

\textbf{Опр.} Доверительным интервалом с уровнем доверия $\gamma$ для параметра $\theta$ называется интервал $(\underline \theta (\vec x), \overline \theta (\vec x))$, отвечающий выборочным значениям статистики $\underline \theta$, $\overline \theta$, задающих оценку уровня $\gamma$ для $\theta$.

\section{Формулы для вычисления границ $\gamma$-доверительного интервала для математического ожидания и дисперсии нормальной случайно величины}

Пусть $X \sim N(\mu, \sigma^2)$, где $\mu$ и $\sigma^2$ --- неизвестны.

Тогда для построения $\gamma$-доверительного интервала для $\mu$ используется
центральная статистика

$$
	\label{eq:T_1}
	T(\vec X, \mu) = \frac{\mu - \overline X}{S(\vec X)} \sqrt n \sim St(n - 1),
$$

\noindentи границы $\gamma$-доверительного интервала для $\mu$ вычисляются по
формулам:

$$
	\label{eq:mu_lower}
	\underline \mu (\vec X) = \overline X - \frac{S(\vec X)t^{(n-1)}_{\frac{1 +
				\gamma}{2}}}{\sqrt n},
$$

$$
	\label{eq:mu_upper}
	\overline \mu (\vec X) = \overline X + \frac{S(\vec X)t^{(n-1)}_{\frac{1 +
				\gamma}{2}}}{\sqrt n},
$$


\noindentгде $\overline X = \frac{1}{n} \sum\limits_{i=1}^{n} X_i$, $S(\vec X) = \sqrt{\frac{1}{n-1} \sum\limits_{i=1}^{n} (X_i - \overline X)^2}$, $t_{\frac{1+\gamma}{2}}^{(n-1)}$ --- квантиль уровня $\frac{1+\gamma}{2}$ распределения Стьюдента с \mbox{$n-1$~степенями} свободы, $n$ --- объем выборки.

Для построения $\gamma$-доверительного интервала для $\sigma^2$ используется
центральная статистика

$$
	\label{eq:T_2}
	T(\vec X, \sigma^2) = \frac{(n-1)S^2(\vec X)}{\sigma^2} \sim \chi^2(n - 1),
$$

\noindentи границы $\gamma$-доверительного интервала для $\sigma^2$ вычисляются по
формулам:

$$
	\label{eq:S_quad_lower}
	\underline \sigma^2 (\vec X) = \frac{(n-1)S^2(\vec X)}{h_{\frac{1+\gamma}{2}}^{(n-1)}},
$$

$$
	\label{eq:S_quad_upper}
	\overline \sigma^2 (\vec X) = \frac{(n-1)S^2(\vec X)}{h_{\frac{1-\gamma}{2}}^{(n-1)}},
$$


\noindentгде $n$ --- объем выборки, $S^2(\vec X) = \frac{1}{n-1} \sum\limits_{i=1}^{n} (X_i - \overline X)^2$, $h_{\frac{1+\gamma}{2}}^{(n-1)}$ и $h_{\frac{1-\gamma}{2}}^{(n-1)}$ --- квантили уровня $\frac{1+\gamma}{2}$ и $\frac{1-\gamma}{2}$ соответственно распределения хи-квадрат с \mbox{$n-1$~степенями} свободы.