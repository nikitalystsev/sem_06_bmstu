\chapter{Теоретические сведения}

\section{Формулы для вычисления величин $M_{max}$,  $M_{min}$, $R$, $\hat{\mu}$, $S^2$}

Пусть $\vec{x} = (x_1, ..., x_n)$ --- выборка объема $n$ из генеральной совокупности $X$.

\begin{itemize}
	\item $M_{min}$ --- минимальное значение выборки $\vec{x}$, определяется по формуле~(\ref{eq:Mmin}).
	
	\begin{equation}
		\label{eq:Mmin}
		M_{min} = x_{(1)} = \min\{x_1, ..., x_n\}
	\end{equation}
	
	\noindentгде $x_{(1)}$ --- крайний левый член вариационного ряда выборки $\vec{x}$.
	
	\item $M_{max}$ --- максимальное значение выборки $\vec{x}$, определяется по формуле~(\ref{eq:Mmax}).
	
	\begin{equation}
		\label{eq:Mmax}
		M_{max} = x_{(n)} = \max\{x_1, ..., x_n\}
	\end{equation}
	
	\noindentгде $x_{(n)}$ --- крайний правый член вариационного ряда выборки $\vec{x}$.
	
	\item $R$ --- размах выборки, определяется по формуле~(\ref{eq:R}).
	
	\begin{equation}
		\label{eq:R}
		R = M_{max} - M_{min}
	\end{equation}
	
	\item $\hat{\mu}$ --- оценка математического ожидания(выборочное среднее), определяется по формуле~(\ref{eq:mu}).
	
	\begin{equation}
		\label{eq:mu}
		\hat{\mu}(\vec{x}) = \frac{1}{n}\sum_{i=1}^{n} x_i.
	\end{equation}
	
	
	\item $S^2$ --- оценка дисперсии(исправленная выборочная дисперсия), определяется по формуле~(\ref{eq:s2}).
	
	\begin{equation}
		\label{eq:s2}
		S^2(\vec x) = \frac 1{n-1} \sum_{i=1}^n (x_i-\overline x)^2,
	\end{equation}
\end{itemize}

\section{Определение эмпирической плотности и гистограммы}

Если объем выборки велик ($n > 50$), то данные группируют не только в виде статистического ряда, но и в виде интервального статистического ряда. 
Для этого выбирается число $m \in \mathbb{N}$ --- количество интервалов, а отрезок $J=[x_{(1)}, x_{(n)}]$ разбивают на $m$ равновеликих промежутков. 
Длина $\Delta$ каждого из них определяется по формуле ~(\ref{eq:delta}).

\begin{equation}\label{eq:delta}
	\Delta = \frac{|J|}{m} = \frac{x_{(n)} - x_{(1)}}{m}.
\end{equation}

Интервалы определяются равенствами (\ref{eq:Ji}).

\begin{equation}
	\label{eq:Ji}
	\begin{aligned}
		J_i &= [x_{(1)} + (i - 1) \Delta; x_{(1)} + i \Delta],
		& i = \overline{1, m-1},\\
		J_m &= [x_{(1)} + (m - 1) \Delta; x_{(n)}].
	\end{aligned}
\end{equation}

\textbf{Опр.} Интервальным статистическим рядом, отвечающим
выборке $\vec{x}$, называется таблица вида:

\begin{table}[htb]
	\centering
	\begin{tabular}{|c|c|c|c|c|}
		\hline
		$J_1$ & ... & $J_i$ & ... & $J_m$ \\
		\hline
		$n_1$ & ... & $n_i$ & ... & $n_m$ \\
		\hline
	\end{tabular}
\end{table}

Здесь $n_i$ --- число элементов выборки $\vec{x}$, попавших в промежуток $J_i$, $i= \overline{1,m}$.

При выборе числа промежутков используют формулу (\ref{eq:m}).

\begin{equation}\label{eq:m}
	m = [\log_2 n] + 2.
\end{equation}

Пусть для данной выборки $\vec{x}$ построен интервальный статистический ряд.

\textbf{Опр.} Эмпирической плотностью распределения соответствующей
выборке $\vec{x}$ называется функция:

\begin{equation}
	f_n(x) =
	\begin{cases}
		\frac{n_i}{n \cdot \Delta}, &x \in J_i,~i = \overline{1, m}, \\
		0, &\text{иначе}. \\
	\end{cases}
\end{equation}

\textbf{Опр.} График эмпирической функции плотности называется гистограммой.

\section{Определение эмпирической функции распределения}

Пусть $\vec{x} = (x_1, ..., x_n)$ --- выборка из генеральной совокупности $X$.

Обозначим $l(t, \vec{x})$ --- число компонент $\vec{x}$, которые меньше, чем $t$ ($t \in \mathbb{R}$).

\textbf{Опр.} Эмпирической функцией распределения, отвечающей выборке $\vec{x}$, называется отображение $F_n: \mathbb{R} \to \mathbb{R}$, заданное формулой: 

\begin{equation}
	F_n(t) = \frac{l(t, \vec x)}{n}.
\end{equation}