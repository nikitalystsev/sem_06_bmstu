\chapter{Исследовательский раздел}

В данном разделе будут приведены постановка исследования и сравнительный анализ алгоритмов на основе полученных данных.

\section{Технические характеристики}

Технические характеристики устройства, на котором проводились исследования: 

\begin{itemize}[label=--]
	\item операционная система: Ubuntu 22.04.3 LTS x86\_64 \cite{info_os};
	\item оперативная память: 16 Гб;
	\item процессор: 11th Gen Intel® Core™ i7-1185G7 @ 3.00 ГГц × 8, 4 физических ядра, 8 логических ядер.
\end{itemize}	

\section{Проведение исследования}

Определим лучший и худший случаи в алгоритме поиска в ДДП и в АВЛ-дереве:

\begin{itemize}[label*=--]
	\item \texttt{лучший случай}, когда искомый элемент находится в корне дерева;
	\item \texttt{худший случай}, когда искомый элемент находится в листе на максимальной высоте дерева, либо случай отсутствия искомого элемента в дереве.
\end{itemize}

\section*{Цель исследования}

Целью исследования является проведение сравнительного анализа количества сравнений, необходимых для решения задачи поиска в ДДП и АВЛ-дереве в лучшем и худшем случае и обоснование выбора худшего случая в программной реализации.

\clearpage 

\section*{Наборы варьируемых и фиксированных параметров}

Замеры времени проводились для числа узлов в дереве, равном 128, 256, 512, 1024, 2048.

В качестве фиксированного параметра для значения, хранящегося в узле бинарного дерева, был выбран разброс от 0 до 2000.

\section*{Результаты исследования}

На рисунке \ref{img:research1} изображены результаты исследования для лучшего случая.

\includesvgimage
{research1} % Имя файла без расширения (файл должен быть расположен в директории inc/img/)
{f} % Обтекание (без обтекания)
{h} % Положение рисунка (см. figure из пакета float)
{1\textwidth} % Ширина рисунка
{Результаты сравнения алгоритма поиска в ДДП и АВЛ-дереве (лучший случай)} % Подпись рисунка

\clearpage

На рисунке \ref{img:research2} изображены результаты исследования для худшего случая, когда искомый элемент находится в листе на максимальной высоте дерева.

\includesvgimage
{research2} % Имя файла без расширения (файл должен быть расположен в директории inc/img/)
{f} % Обтекание (без обтекания)
{h} % Положение рисунка (см. figure из пакета float)
{1\textwidth} % Ширина рисунка
{Результаты сравнения алгоритма поиска в ДДП и АВЛ-дереве (худший случай)} % Подпись рисунка
\clearpage

На рисунке \ref{img:research3} изображены результаты исследования для худшего случая, когда искомый элемент отстутствует в дереве.

\includesvgimage
{research3} % Имя файла без расширения (файл должен быть расположен в директории inc/img/)
{f} % Обтекание (без обтекания)
{h} % Положение рисунка (см. figure из пакета float)
{1\textwidth} % Ширина рисунка
{Результаты сравнения алгоритма поиска в ДДП и АВЛ-дереве (худший случай 2)} % Подпись рисунка

\section*{Вывод}

Число сравнений в лучшем случае всегда равно единице и не зависит от числа узлов в дереве, так как алгоритм поиска начинает свою работу с корня дерева, производит первое сравнение и находит нужный результат.

С ростом числа узлов растет максимальная высота дерева, и, следовательно растет число сравнений для худшего случая, когда искомый элемент находится в листе на максимальной высоте дерева.

\clearpage 

Для случая, когда искомый элемент отсутствует в дереве, число сравнений зависит структуры дерева. Для АВЛ-деревьев среднее число сравнений остается относительно невысоким, так как высота дерева ограничена.

Таким образом, в данной программной реализации алгоритма поиска целого числа в ДДП и АВЛ-дереве
худшим случаем считается такой, когда искомый элемент находится в листе на максимальной высоте дерева.



