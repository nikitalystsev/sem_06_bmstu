\chapter{Конструкторский раздел}

\section{Алгоритм поиска}

Определим следующие операторы и функции:

\begin{itemize}[label*=--]
	\item оператор $\gets$ обозначает присваивание значение переменной;
	\item функция $root(T)$ возвращает узел --- корень дерева $T$;
	\item функция $key(x)$ возвращает ключ узла $x$;
	\item функция $left(x)$ возвращает узел --- левое поддерево узла $x$;
	\item функция $right(x)$ возвращает узел --- правое поддерево узла $x$.
\end{itemize}

На рисунке \ref{alg:alg_01} представлен псевдокод рекурсивного алгоритма поиска узла по ключу в ДДП и АВЛ-дереве.

\begin{algorithm}[H]
	\caption{Псевдокод рекурсивного алгоритма поиска узла по ключу в ДДП и АВЛ-дереве}
	\label{alg:alg_01}
	\text{\textbf{На входе}: дерево $T$, $k$ --- значение ключа}\\
	\begin{algorithmic}[1]
		\Function{TreeRecSearch}{$T$, $k$}
		\State $x \gets root(T)$
		
		\If{$x = NULL$ \textbf{and} $k = key(x)$}
		\State \Return $x$
		\EndIf
		
		\If{$k < key(x)$}
		\State \Return \Call{TreeRecSearch}{$left(x)$, $k$}
		\Else
		\State \Return \Call{TreeRecSearch}{$right(x)$, $k$}
		\EndIf
		
		\EndFunction
	\end{algorithmic}
\end{algorithm}

%На рисунке \ref{alg:alg_02} представлен псевдокод итеративного алгоритма поиска узла по ключу в ДДП и АВЛ-дереве.
%
%\begin{algorithm}[H]
%	\caption{Псевдокод итеративного алгоритма поиска узла по ключу в ДДП и АВЛ-дереве}
%	\label{alg:alg_02}
%	\text{\textbf{На входе}: дерево $T$, $k$ --- значение ключа}\\
%	\begin{algorithmic}[1]
%		\Function{TreeItSearch}{$T$, $k$}
%		\State $x \gets root(T)$
%		
%		\While{$x \neq NULL$ \textbf{and} $k \neq key(x)$}
%		\If{$k < key(x)$}
%		\State $x \gets left(x)$
%		\Else
%		\State $x \gets right(x)$
%		\EndIf
%		\EndWhile
%		
%		\State \Return $x$
%		
%		\EndFunction
%	\end{algorithmic}
%\end{algorithm}


\section*{Вывод}

В данном разделе были описан псевдокод для алгоритма поиска значения по ключу в несбалансированнном двоичном дереве поиска и в АВЛ-дереве.



