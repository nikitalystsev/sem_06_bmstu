\chapter*{ЗАКЛЮЧЕНИЕ}
\addcontentsline{toc}{chapter}{ЗАКЛЮЧЕНИЕ}

В ходе выполнения лабораторной работы были решены следующие задачи:

\begin{enumerate}[label={\arabic*)}]
	\item описан используемый алгоритм поиска;
	\item выбраны средства программной реализации;
	\item реализован данный алгоритм поиска;
	\item проанализирован алгоритм по количеству сравнений.
\end{enumerate}

Цель данной лабораторной работы, а именно исследование лучших и худших случаев алгоритма поиска целого числа в несбалансированном двоичном дереве поиска (ДДП) и сбалансированном (АВЛ-дереве), также была достигнута.

Число сравнений в лучшем случае всегда составляет единицу и не зависит от количества узлов в дереве. 
Это объясняется тем, что алгоритм поиска начинает свою работу с корня дерева, выполняет первое сравнение и находит требуемый результат.

С увеличением числа узлов в дереве растет максимальная высота дерева, следовательно, возрастает и число сравнений в худшем случае, когда искомый элемент расположен в листе на максимальной высоте дерева.

В случае отсутствия искомого элемента в дереве, количество сравнений зависит от структуры дерева. В сбалансированных деревьях, таких как АВЛ-деревья, среднее число сравнений остается относительно низким, поскольку высота дерева ограничена.

Таким образом, в данной программной реализации алгоритма поиска целого числа в ДДП и АВЛ-дереве, худшим случаем считается ситуация, когда искомый элемент находится в листе на максимальной высоте дерева.
